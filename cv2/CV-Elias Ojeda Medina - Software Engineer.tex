%%%%%%%%%%%%%%%%%
% This is an sample CV template created using altacv.cls
% (v1.3, 10 May 2020) written by LianTze Lim (liantze@gmail.com). Now compiles with pdfLaTeX, XeLaTeX and LuaLaTeX.
%
%% It may be distributed and/or modified under the
%% conditions of the LaTeX Project Public License, either version 1.3
%% of this license or (at your option) any later version.
%% The latest version of this license is in
%%    http://www.latex-project.org/lppl.txt
%% and version 1.3 or later is part of all distributions of LaTeX
%% version 2003/12/01 or later.
%%%%%%%%%%%%%%%%

%% If you need to pass whatever options to xcolor
\PassOptionsToPackage{dvipsnames}{xcolor}

%% If you are using \orcid or academicons
%% icons, make sure you have the academicons
%% option here, and compile with XeLaTeX
%% or LuaLaTeX.
% \documentclass[10pt,a4paper,academicons]{altacv}

%% Use the "normalphoto" option if you want a normal photo instead of cropped to a circle
% \documentclass[10pt,a4paper,normalphoto]{altacv}

\documentclass[10pt,a4paper,ragged2e,withhyper]{altacv}
%% AltaCV uses the fontawesome5 and academicons fonts
%% and packages.
%% See http://texdoc.net/pkg/fontawesome5 and http://texdoc.net/pkg/academicons for full list of symbols. You MUST compile with XeLaTeX or LuaLaTeX if you want to use academicons.

% Change the page layout if you need to
\geometry{left=1.25cm,right=1.25cm,top=1.5cm,bottom=1.5cm,columnsep=1.2cm}

% The paracol package lets you typeset columns of text in parallel
\usepackage{paracol}

% Change the font if you want to, depending on whether
% you're using pdflatex or xelatex/lualatex
\ifxetexorluatex
  % If using xelatex or lualatex:
  \setmainfont{Roboto Slab}
  \setsansfont{Lato}
  \renewcommand{\familydefault}{\sfdefault}
\else
  % If using pdflatex:
  \usepackage[rm]{roboto}
  \usepackage[defaultsans]{lato}
  % \usepackage{sourcesanspro}
  \renewcommand{\familydefault}{\sfdefault}
\fi

% Change the colours if you want to
\definecolor{SlateGrey}{HTML}{2E2E2E}
\definecolor{LightGrey}{HTML}{666666}
\definecolor{DarkPastelRed}{HTML}{1A1919}
\definecolor{PastelRed}{HTML}{343a40}
\definecolor{GoldenEarth}{HTML}{208337}
\colorlet{name}{black}
\colorlet{tagline}{PastelRed}
\colorlet{heading}{DarkPastelRed}
\colorlet{headingrule}{GoldenEarth}
\colorlet{subheading}{PastelRed}
\colorlet{accent}{PastelRed}
\colorlet{emphasis}{SlateGrey}
\colorlet{body}{LightGrey}

% Change some fonts, if necessary
\renewcommand{\namefont}{\Huge\rmfamily\bfseries}
\renewcommand{\personalinfofont}{\footnotesize}
\renewcommand{\cvsectionfont}{\LARGE\rmfamily\bfseries}
\renewcommand{\cvsubsectionfont}{\large\bfseries}


% Change the bullets for itemize and rating marker
% for \cvskill if you want to
\renewcommand{\itemmarker}{{\small\textbullet}}
\renewcommand{\ratingmarker}{\faCircle}

%% sample.bib contains your publications
\addbibresource{sample.bib}

\begin{document}
\name{Elias Ojeda Medina}
\tagline{Software Engineer}
%% You can add multiple photos on the left or right
\photoR{2.8cm}{eocode}
% \photoL{2.5cm}{Yacht_High,Suitcase_High}

\personalinfo{%
  % Not all of these are required!
  \email{}
  \phone{}
  \location{Ciudad de México, México}
  \homepage{eliasojedamedina.com}
  \twitter{@eocode}
  \linkedin{eocode}
  \github{eocode}
  %% You MUST add the academicons option to \documentclass, then compile with LuaLaTeX or XeLaTeX, if you want to use \orcid or other academicons commands.
  % \orcid{0000-0000-0000-0000}
  %% You can add your own arbtrary detail with
  %% \printinfo{symbol}{detail}[optional hyperlink prefix]
  % \printinfo{\faPaw}{Hey ho!}[https://example.com/]
  %% Or you can declare your own field with
  %% \NewInfoFiled{fieldname}{symbol}[optional hyperlink prefix] and use it:
  % \NewInfoField{gitlab}{\faGitlab}[https://gitlab.com/]
  % \gitlab{your_id}
}

\makecvheader
%% Depending on your tastes, you may want to make fonts of itemize environments slightly smaller
% \AtBeginEnvironment{itemize}{\small}

%% Set the left/right column width ratio to 6:4.
\columnratio{0.6}

% Start a 2-column paracol. Both the left and right columns will automatically
% break across pages if things get too long.
\begin{paracol}{2}
\cvsection{Experiencia}

\cvevent{Analista de inteligencia de negocios}{Centro Medico ABC}{2017 - 2019}{Ciudad de México}
\begin{itemize}
\item Colaboración con diferentes áreas en el analisis y procesamiento de información
\item Creación de un datawarehouse corporativo, Desarrollo de algoritmos, visualización de datos en Dashboards, Desarrollo de conectores SAP, 
\item Desarrollo web en intranet para compartir información
\end{itemize}

\divider

\cvevent{Analista de Información estratégica}{Centro Medico ABC}{2016 - 2017}{Ciudad de México}
\begin{itemize}
\item Generación de reportes y Analisis de información
\item Desarrollo de algoritmos
\end{itemize}

\cvsection{Proyectos}

\cvevent{\href{https://lesqui.com/}{Lesqui}}{App móvil para micronegocios}{9 meses tiempo parcial}{Remoto}
\begin{itemize}
	\item Laravel Django PostgreSQL GCP AWS Flutter
\end{itemize}

\divider

\cvevent{\href{https://gitlab.com/teamspartans/rommie}{Kumpel (5 personas)}}{PWA para rentar y ver habitaciones}{2 semanas}{Remoto}
\begin{itemize}
	\item Django Python PostgreSQL GCP GitLab GraphQL React
\end{itemize}

\divider

\cvevent{\href{https://gitlab.com/teamspartans/pokeguia}{PokeguIA (4 personas)}}{Analisis, procesamiento y visualización de pokemons}{2 semanas}{Remoto}
\begin{itemize}
\item Django Python PostgreSQL GCP GitLab GraphQL React
\end{itemize}

\medskip

\cvsection{Un día cómun}

% Adapted from @Jake's answer from http://tex.stackexchange.com/a/82729/226
% \wheelchart{outer radius}{inner radius}{
% comma-separated list of value/text width/color/detail}
\wheelchart{1.5cm}{0.5cm}{%
  5/8em/accent!25/{Dormir,\\Buen descanso},
  6/8em/accent!33/Trabajo,
  2/8em/accent!10/Aprender,
  2/10em/accent!10/Comer,
  2/10em/accent!10/Ejercicio,
  2/6em/accent!20/Proyectos personales,
  3/6em/accent!20/otros
}

%% Switch to the right column. This will now automatically move to the second
%% page if the content is too long.
\switchcolumn

\cvsection{Filosofía de vida}

\begin{quote}
``La imaginación es el principio de la creación.  Imaginas lo que deseas, persigues lo que imaginas y finalmente, creas lo que persigues.'' Bernard Shaw
\end{quote}

\cvsection{Habilidades}

\cvtag{Organización}
\cvtag{Trabajo en equipo}\\
\cvtag{Proactivo}
\cvtag{Liderazgo}
\cvtag{Creativo} \\
\cvtag{Resiliencia}

\divider\smallskip

\cvtag{Python}
\cvtag{Django} 
\cvtag{Flask}
\cvtag{SQL} \\
\cvtag{GCP}
\cvtag{AWS}
\cvtag{Docker}
\cvtag{GitLab}

%% Yeah I didn't spend too much time making all the
%% spacing consistent... sorry. Use \smallskip, \medskip,
%% \bigskip, \vpsace etc to make ajustments.
\medskip

\cvsection{Educación}

\cvevent{\href{https://platzi.com/@eocode/}{Platzi Master}}{}{2016 - Actualidad}{}
Data Science, Backend, Frontend, DevOps, Cloud

\divider

\cvevent{Ingenieria en Computación}{Instituto Politécnico Nacional (IPN/ESIME)}{Ago 2013 -- Ago 2017}{}
Especialidad: Sistemas de información

\cvsection{Reconocimientos}

\cvachievement{\faTrophy}{\href{https://eliasojedamedina.com/scrum.pdf}{Scrum SFPC}}{CERT ID: 41900668.}

\divider

\cvachievement{\faTrophy}{\href{https://www.youracclaim.com/users/eocode/badges}{Certificaciones Java}}{CERT ID:
	243138007OCPJSE6P y CERT ID: 243138007OCAJSE7}

\divider

\cvachievement{\faHeartbeat}{Programa de formación de líderes}{Inroads de México}

\cvsection{Idiomas}

\cvskill{Español}{5}
\cvskill{Inglés}{2}
\divider

\end{paracol}


\end{document}
